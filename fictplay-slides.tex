\documentclass[dvipdfmx,fleqn]{beamer}
%\documentclass[dvipdfmx,fleqn,handout]{beamer}
\usepackage{amsmath,amssymb,amsthm}

\mode<presentation>
{
  \usetheme{default}
}

\title{\Large タイトル}
\author{\large 名前}
\date{\small 日付}

\usefonttheme{professionalfonts}

\setbeamercovered{transparent=20}

\setbeamertemplate{navigation symbols}{} 
\setbeamertemplate{footline}[frame number] 



\begin{document}

\sffamily
\gtfamily


\begin{frame}
  \titlepage
  \thispagestyle{empty}
\end{frame}

\setcounter{framenumber}{0}




\begin{frame}
\frametitle{はじめに}
\begin{itemize}\setlength{\parskip}{0.5em}
\item
項目1

\item
項目2
 \begin{itemize}\setlength{\parskip}{0.5em}
 \item
 1階層下の項目1
 \item
 1階層下の項目2
 \end{itemize}

\item
このページの最後の項目
\end{itemize}
\end{frame}



\begin{frame}
\frametitle{次のスライド}
\begin{itemize}\setlength{\parskip}{0.5em}
\item
Ficititious playの説明1

\item
Ficititious playの説明2 \pause

$x_0(t)$ は
\[
x_0(t+1)
= x_0(t) + \frac{1}{t+2} (a_1(t) - x_0(t))
\]
と再帰的に書くことができる. \pause

\item
Ficititious playの説明3 \pause

\item
``\texttt{pause}''をつけるとoverlayができる.

\item
ファイルの冒頭の\texttt{documentclass}のオプションで\texttt{handout}を指定すると,
overlayにならずいっぺんに表示される.

Webにのせるときや,印刷して配るときなどは\texttt{handout}を指定しておく.

\end{itemize}
\end{frame}



\begin{frame}[containsverbatim]% verbatim 環境を使えるように
\frametitle{コードの説明とか}
\begin{itemize}\setlength{\parskip}{0.5em}
\item
コードの表示の例
\begin{verbatim}
import numpy
from matplotlib import pyplot

x = numpy.arange(0, 10, 0.1)
y = numpy.cos(x)
pyplot.plot(x,y)
pyplot.show()
\end{verbatim}

\item
\verb|\begin{frame}| から \verb|\end{frame}| までを
コピー\&ペーストしてスライドを増やしていく.
\end{itemize}
\end{frame}



\begin{frame}
\frametitle{図}
\begin{figure}
 \centering
 \includegraphics[scale=0.3]{matchingpennies_plot.pdf}
 \caption{図の表示}
 \label{fig:matchingpennies_plot}
\end{figure}
\end{frame}



\begin{frame}
\frametitle{まとめ}
\begin{itemize}\setlength{\parskip}{0.5em}
\item
まとめ

\item
よくわかっていない点とか

\item
今後の課題とか
\end{itemize}
\end{frame}



\end{document}